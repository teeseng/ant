\documentclass[12pt]{amsart}
\usepackage[utf8]{inputenc} 
\usepackage[top=2cm, bottom=2cm, left=2cm, right=2cm]{geometry}
\usepackage{indentfirst}
\usepackage{titlesec}
\usepackage{scrextend}
\usepackage{amsmath}
\usepackage{amssymb}
\usepackage{array}

\title{Generalization and Implementation of Paillier Protocol}

\author{Andrew Tseng: art2589}
\begin{document}
\maketitle

\begin{abstract}
    Electronic voting has been tested throughout by countries like Estonia and
    security has been a main issue. We look into the general characteristics of
    the Paillier protocol and how it is relevant to electronic voting in general.
    There is also an implementation of the Paillier protocol in Python 2.7 along
    with tests for its correctness and looks for vulnerabilities in the cryptographic
    system. We talk about the performance of the implementation and the provide
    explicit documentation to the code written in gen.py, encrypt.py, decrypt,py.\\
\end{abstract}

\section*{\textbf{Introduction to Paillier Protocol}}

\indent Electronic voting utilizes various forms of cryptographic protocols and
one of them is the Paillier protocol. The Paillier protocol is a probabilistic
encryption scheme that generates keys off computation of random values from the 
group $Z_{n}^{*}$. The encryption process of Paillier results from generating a
random integer and using the RSA modulus of $n$. A favorable characteristic of
the encryption process is the homomorphic flavor of this protocol which makes it
favorable for electronic voting. The security of the Paillier protocol relies
on the fact that it is difficult to figure out if two elements lie within the
same coset or not.

\section*{\textbf{Benefits to Paillier}}

\indent The private key for a cryptographic system must have a hash function that
can create signatures that do not duplicate across voters throughout the large
population of participants. The Paillier protocol requires a finding two large prime
numbers to generate a $n$ in order to create a large subset of values to generate
public and private keys during the key-gen phase. \\

\indent Homomorphism allows for identity values with signatures to correlate to
tallying while still encrypted. Say Alice sends cypher $c_{1}$ for $m_{1}$ and
Bob sends cypher $c_{2}$ for $m_{2}$, then the cypher $c_{1}c_{2}$ is an encryption
of $m_{1}m_{2}$. This allows for candidates to be tallied by the signature of 
the voter and prevents the public from seeing who
voted for who, which can be controversial. From the standpoint of the key-gen
phase, it is very difficult to get the cipher text $c$, choice of $g$ as this
pair was randomly generated.

\indent This allows the encrypted identity of the voters to stay encrypted when it comes
to couting the number of votes for the candidates. The protocol is oftenly named
one-way as a result of it.\\

\indent Because Paillier is a public key encryption scheme, the need to send the
cypher to more than twice as knowing the cypher does not help with decrypting
the message. Sending $c$ and $c'$ wouldn't decrypt to the same message because of
the scheme and $c$ will blind itself from the original message $m$.

\section*{\textbf{Explanation of Protocol}}

\textbf{Key Generator} - The public and private keys are generated during this
phase by choosing a random $p,q$ such that they are independent and prime from
eachother. The modulo $n = pq$ where the RSA modulos is set for the protocol. 
One way to ensure these two prime numbers are independent is to make sure that
$p,q \in {0,1}^{s}$ meaning they have equivalent bit length, and prime but not
equal. The goal of the key-gen phase is to generate $(n,g), (\lambda, u)$. \\

\textbf{Encryption Phase} - Given message $m$, a randomly generated $r \in
Z_{n}^{*}$, and the public key $(n,g)$. The cipher $c = g^{m} * r^{n} \mod n^2$
is the encryption of $m$. \\

\textbf{Decryption Phase} Given cipher $c$ and the private key $(\lambda, u)$
We decrypt $m = (L(c^{\lambda} \mod n^2) * u) \mod n$. A way to increase the
performace of the decryption is the calculation of the $L$ of the function. \\

 The decryption process involves solving a discrete log problem by taking cipher
text $c$ and computing it by $m = L(c^{\lambda} \mod n^2) * eu \mod n)$, where
\[ L(x) = \frac{x-1}{n} \]
\[\lambda = lcm(p-1, q-1)\]

\section*{\textbf{Documentation of Implementation}}

\noindent \textbf{gen.py} generates the private key $(\lambda,u)$ and public key
$(n,g)$. \\ The private key is used for the decryption process while the public is used
for encryption. (decrypt.pyt and encrypt.py)\\

\indent Function \textbf{gen\_ab} generates $\alpha, \beta$ which are used to
randomly generate $g$ under the group. \\

\indent Function \textbf{gen\_g} is a second way to generate a $g \in Z_{n^2}^{*}$
where $\gcd(\frac{g^{\lambda} \mod (n^2 - 1)}{n}, n) = 1$ \\

\indent Function \textbf{calc\_u} calculates the $u$ used for the private key given
$g,n,\lambda$. \\

\noindent \textbf{encrypt.py} Given message $m$ $r$ is selected randomly from
the group $r \in Z_{n}^{*}$ The ciphertext $c = g^{m} * r^{n} \mod n^2$, where
$g$ is randomly selected from $Z_{n^2}^{*}$ and suffices the following requirement:
$\gcd(\frac{g^{\lambda} \mod (n^2 - 1)}{n}, n) = 1$ The program takes in a $c, \lambda,
u,n$ in that order from input.txt\\

\noindent \textbf{decrypt.py} Given ciphertext $c \in Z_{n^2}^{*}$, the message

$m = L(c^{\lambda} \mod n^2) * u \mod n$. \\


\section*{\textbf{Weaknesses and Attacks}}
Producing a small $p,q$ where the values are small can lead to the easy finding of
$n$ which destroys the homomorphism of the protocol. The reason for this is from
finding $n$ would reduce solving $c$ to a easily factoring $n$. Increasing
the lower bound that $p,q$ can be generated to from the random function will prevent
the $n$ from being too small for the protocol to encrypt the message. \\

\section*{\textbf{Performance of the Implementation}}
As a result of the random generating $p,q$ and to have them be prime and independent
from one another, the time it takes for the key generator to execute and produce
the public key and private key varies greatly on how quickly \textbf{gen\_pq} function
can find the pair $(p,q)$ that satisfies the requirement.\\


Varying results of the runtime of gen.py specifically.\\
\begin{center}
\begin{tabular}{|c|}
 \hline
 84.1 sec \\
 7.43 sec \\
 27.9 sec\\
 70.9 sec \\
 \hline
\end{tabular}
\end{center}

The time it takes to find a pair of $p,q$ depends on how quickly the random
number generator generates a pair of prime numbers with the same bit length
such that $p,q \in \{0,1\}^{s-1}$ where $s$ is the bit length.

\newpage
\section*{\textbf{Reference}}
Jurik, Mads and Ivan Damgard, \textit{A Generalization of Paillier’s Public-Key System
with Applications to Electronic Voting}, Aarhus University. \\

Hoffstein, Jeffrey and Pipher, Jill and Silverman, Joseph H, \textit{An Introduction to
Cryptography, 2nd Edition}, Springer 2014. \\


\end{document}

