\documentclass[10pt]{amsart}
\usepackage[utf8]{inputenc}
\usepackage[margin=0.50in]{geometry}
\usepackage{indentfirst}
\usepackage{graphicx}
\usepackage{amsthm}
\usepackage{scrextend}
\usepackage{amssymb}
\usepackage{amsmath}
\usepackage{titlesec}

\title{\LARGE M343L: Homework Set 8 Proofs}
\author{Andrew Tseng: art2589}
\begin{document}
\maketitle
\thispagestyle{empty}

\section*{\small \textbf{Problem 6.17}}

\noindent \textbf{\small Part A:} \\
To prove $m'_{1} = m_{1}, m'_{2} = m_{2}$. \\
\textit{Proof:}
We know that $S = n_{1}R = T$ where $S = kQ_{a}, R = kP$ which makes shows
that the pairing will lead to
\[x_{T}^{-1}x_{S}m_{1} = m'_{1} = m_{1}\]
\[y_{T}^{-1}y_{S}m_{2} = m'_{2} = m_{2}\]

\noindent \textbf{\small Part B:} \\
The message given from MV-elgamal encryption is $(R, c_{1}, c_{2})$ \\

\noindent \textbf{\small Part C:} \\
Alice Encryption Key: $Q_{A} = (1104, 492)$.
$(m_{1}, m_{2}) = (509, 980)$\\

\section*{\small \textbf{Problem 6.18}}
\noindent \textbf{\small Part A:}
Since Eve knows $E$ and we know that $c_{1} = x_{P}m_{1}, c_{2} = y_{P}m_{2}$,
thus $x_{P} = \frac{c_{1}}{m_{1}}, y_{P} = \frac{c_{2}}{m_{2}}$. Plugging these
values into the curve:
\[(\frac{c_{2}}{m_{2}})^2 = (\frac{c_{1}}{m_{1}})^3 + A(\frac{c_{1}}{m_{1}})
+ B\]

Eve then can solve for the roots of this polynomials for $m_{1}$ or $m_{2}$,
given she knows one of these values. It becomes as simple as solve for the roots.\\

\noindent \textbf{\small \textbf{Part B:}}
Since $\frac{814}{1050} \in F_{1201} = 957$
Given the previous $E$, we can find two possible solutions to
\[y_{s}^2 = (957)^3 + 19(957) + 17\]
\textit{$y \in [182, 1019]$}\\
If $y = 182$, then the message pair is (1050,440).
If $y = 1019$, then the message pair is (1050, 761).\\

\section*{\small \textbf{Problem 6.29}}
\textit{Proof: Given $R(x),S(x)$ are rational functions}
Since $div(f) = \sum_{Z}^{} ord(f)Z$, then due to the additive properties of the functions
at the points of the curve, we can say that 
\[div(R(x)S(x)) = div(R(x)) + div(S(X))\]
\\

\section*{\small \textbf{Problem 6.32}}


\section*{\small \textbf{Problem 6.33}}


\end{document}

