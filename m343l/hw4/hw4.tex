\documentclass[10pt]{amsart}
\usepackage[utf8]{inputenc} 
\usepackage[margin=0.50in]{geometry}
\usepackage{indentfirst}
\usepackage{graphicx}
\usepackage{amsthm}
\usepackage{scrextend}
\usepackage{amssymb}
\usepackage{array}

\title{Proofs: Homework 4}
\author{Andrew Tseng: art2589}
\begin{document}
\maketitle
\thispagestyle{empty}

\section*{\large \textbf{Problem 2.34}}
\begin{addmargin}{5pt}
\textbf{\small Part A} \\
Given that $a$ and $b$ are nonzero polynomials, then the $\deg(a)$ and $\deg(b)$
are the highest power within the polynomials. By definition and properties of
multiplying, $a \cdot b$ is the field of the product of the two nonzero
polynomials which means that highest power of $a \cdot b$ is the $\deg(a)
+ \deg(b)$.  \\

\noindent \textbf{\small Part B} \\
Assume that polynomial $a$ has a multiplicative inverse, meaning that there
exists polynomials $b,c \in F[x]$. Thus $b(x)c(x) = 1$, since $b,c$ exist in
$F[x]$, then $\deg(b) = \deg(c) = 0$. Meaning that $\deg(a) = 0$, implying
that $a$ is a constant polynomial. \\

\noindent Assume that the polynomial $a$ is a constant polynomial, meaning that 
$\deg(a) = 0$. Let $b,c$ be the multiplicative inverses of $a$ such that
$c(x)b(x) = 1$. Since $\deg(b) + \deg(c) = 0$ and we know that $b,c$ are
nonnegative polynomials since they are in the field $F$, then $b,c$ are
multiplicative inverses. \\

\noindent \textbf{\small Part C} \\

\noindent \textbf{\small Part D} \\

\end{addmargin}

\section*{\large \textbf{Problem 2.37}} 
\begin{addmargin}{5pt}   
Reducing the polynomial to the form, $(x + a)(x + b)(x + c)$, we create the 
system equations with $a,b,c$
\[abc           = 1\]
\[2(a + b) + c  = 0\]
\[ab + 2c(a+b)  = 1\]
\\
After solving the variables we find that $a + b + c = 0$ and $abc = 1$, where
there is no solution since $a,b,c > 0$. Thus the polynomial is irreducuble. \\
\end{addmargin}

\section*{\large \textbf{Problem 2.38}}
\begin{addmargin}{5pt}    
\begin{center}
 \begin{tabular}{||c c c c c c c ||} 
 \hline
$1$ & $x$ & $x^2$ & $1 + x$ & $1 + x^2$ & $x + x^2$ & 
    $1 + x + x^2$ \\ [0.5ex] 
 \hline\hline
  0 & 0 & 0 & 0 & 0 & 0 & 0 \\ 
 \hline
 $1$ & $x$ & $x^2$ & $1 + x$ & $1 + x^2$ & $x + x^2$ & $1 + x + x^2$  \\ 
 \hline
 $x$   & $x^2$ & $x + 1$ & $x + x^2$ & $1$ & $x^2 + x + 1$  & $x^2 + 1$\\
\hline
 $x + 1$   & $x^2 + x$ & $x^2 + x + 1$ & $1 + x^2$ & $x^2$  & $1$  & $x$\\
 \hline
 $x^2$ & $x+1$ & $x + x^2$ & $1 + x + x^2$ & $1 + x^2$ & $1$ & $x$ \\
 \hline
 $1 + x^2$ & $1$ & $x$  & $x^2$ & $ 1 + x + x^2 $ & $1 + x$ & $x + x^2$  \\ [0.5ex] 
 \hline
$x + x^2$ & $x^2 + x + 1$ & $1 + x^2$ & $1$ & $1 + x$ & $x$ & $x^2$  \\ 
 \hline
 $1 + x + x^2$ & $1 + x^2$ & $1$  & $x$ & $x + x^2$  & $x^2$  & $1 + x$  \\ [0.5ex] 
\hline \\
\end{tabular}
\end{center}
\end{addmargin}


\section*{\large \textbf{Problem 2.40}}
\begin{addmargin}{5pt}    
Both rings hold $p^e$ elements but $F_{p^e}[X]$ holds the coefficients of 
the polynomial while $Z/(p^e)Z$ holds all values from $0$ to $p^e - 1$. This 
means all values in the modulos ring are distinct while the polynomial ring 
does not guarantee this.  
\end{addmargin}


\end{document}

