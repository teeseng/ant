\documentclass[10pt]{amsart}
\usepackage[utf8]{inputenc}
\usepackage[margin=0.50in]{geometry} \usepackage{indentfirst}
\usepackage{graphicx}
\usepackage{amsthm}
\usepackage{scrextend}
\usepackage{amssymb}
\usepackage{titlesec}

\title{M343L: Homework Set 6 Proofs}
\author{Andrew Tseng: art2589}
\begin{document}
\maketitle
\thispagestyle{empty}

\section*{\large \textbf{Problem 4.8}}

\noindent \textbf{\small Part A} \\
At a glance, Eve can check if $S_{1} = S_{1}'$ to check if Samantha used the
same $k$ to sign $D,D'$. \\ This is because within both of the processes of
signing the two documents all have the same $g,p,a$ in the Elgamal Signature. \\

\noindent \textbf{\small Part B} \\
You can first solve for $a$ with the given  $S_{2}, S_{2}'$.
\[k(S_{2} +S_{2}') = (D + D') - a(S_{1} + S_{1}')\]
$k$ is found by solving the DLP of $g^k = S_{1} \mod p$ (using Shanks).
All the calculations for solving for $a$ is done in $F_{p}$.
\\

\noindent \textbf{\small Part C} \\
Solve the DLP for the $k$.\\
$k = 1$.
Plugging in the values we find $a = 348145$.\\

\section*{\large \textbf{Problem 5.30}}
\noindent$E = n$ is prime, $F = $ the Miller-Rabin test fails $N$ times \\
The MR-test always fails when $n$ is prime, and the rate 
$\Pr(E) = \frac{1}{\ln(n)}, \Pr(F|E^c) = \frac{1}{4^n}$ \\
It is clear that $\Pr(F | E) = 1$, since if $n$ is prime, then the Miller-Rabin
test fails no matter how many times.\\

Using the Monte-Carlo Algorithm:
\[\Pr(E | F) = \frac{\Pr(F|E)\Pr(E)}{\Pr(F|E)\Pr(E) + \Pr(F|E^c)\Pr(E^c)} \]
\[ = \frac {\frac{1}{\ln(n)}} {4^{-N}(1 - \frac{1}{\ln(n)}) + \frac{1}{\ln(n)}} \]
\[ = 1 - \frac{\ln(n) - 1}{4^N + \ln(n) - 1} > 1 - \frac{\ln(n)}{4^N}\]
\\

\section*{\large \textbf{Problem 5.38}}

\noindent \textbf{\small Part A} \\
Taking the second deriviative of
$f(x) = e^{-x} - (1-x)$. \\
Finding the zeroes of $f'(x) = -e^{-x} + 1$, we get that $x=0$. Meaning that
$f(0)$ is the minimum of $f(x)$ which we find to be $0$. Thus for all $x$,
\[e^{-x} \geq 1 - x\]

\noindent \textbf{\small Part B} \\
We use the same technique from part A with the second derivative with
$f(x) = -e^{-ax} + (1-x)^a + \frac{1}{2}ax^2$ \\
We find that the min is again $0$ and is at the end point. Thus it is clear that
for all $x$, $f(x) \geq 0$. \\

\noindent \textbf{\small Part C} \\
Let $a=m, x = \frac{n}{N}$. \\
The probability to get at least one red:
\[\Pr(E) = 1 - (1 - \frac{n}{N})^m\]

From part b:
\[1 - e^{\frac{nm}{N}} \geq 1 - (1 - \frac{n}{N})^m - \frac{mn^2}{2N^2}\]
Moving and isolating the sides:
\[ 1 - (1 - \frac{n}{N}^m) \leq 1 - e^{\frac{nm}{N}} + \frac{mn^2}{2N^2}\]

We conclude:
\[\Pr(E) \leq 1 - e^{\frac{nm}{N}} + \frac{mn^2}{2N^2}\]

Given that $N$ and $n$ is small relative to $N$, then we know that
$\frac{mn^2}{2N^2}$ converges to zero as $N$ grows larger and $n$ stays small.
While $\frac{-mn}{N}$ also converges to 0 but not as fast as the previous expression.
Thus at some range where $N$ is large,
\[\Pr(E) \leq 1 - e^{\frac{-mn}{N}}\]
\\

\section*{\large \textbf{Problem 5.43}}
Calculating $I^2$ and converting it to polar.
\[ a = \int_{0}^{2\pi}\cos^2\theta\sin^2\theta d\theta = 4\pi\]
\[ b = \int_{0}^{\infty}r^5e^{\frac{-r^2}{2}}dr = 1\]
\[\sqrt{ab} = I = 2\sqrt{\pi}\]

\end{document}
