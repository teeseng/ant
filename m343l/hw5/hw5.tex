\documentclass[10pt]{amsart}
\usepackage[utf8]{inputenc}
\usepackage[margin=0.50in]{geometry} \usepackage{indentfirst}
\usepackage{graphicx}
\usepackage{amsthm}
\usepackage{scrextend}
\usepackage{amssymb}

\title{Proofs: Homework 5}
\author{Andrew Tseng: art2589}
\begin{document}
\maketitle
\thispagestyle{empty}

\section*{\large \textbf{Problem 3.5}}
\begin{addmargin}{8pt}

\textbf{\small Part A} \\
If $p$ and $q$ are distinct primes, then $\phi(pq) = \phi(q) * \phi(p)$. \\

\noindent \textbf{\small Part B} \\
If $p$ is prime, then $\phi(p^2) = p - 1$. \\
We will prove that: \\
If $p$ is prime, then $\phi(p^j) = p^{j} - p^{j-1}$. \\
Let $m$ be a number that is less than $p^k$, the only way $\gcd(m,p^k) > 1$ if
$m$ is a multiple of $p$. Through since there are a number of $p^{j-1}$ multiples
in a range of $1$ to $p^j$. Thus the number of $m$ that have suffice with the
requirements of the phi function is $p^j - p^{j-1}$.

Formula was analyzed from running many results from phi.py \\

\noindent \textbf{\small Part C} \\
Since $\gcd(M,N) = 1$, then $M,N$ are distinct primes, which proves
from part A that $\phi(MN) = \phi(M)\phi(N)$. \\

\noindent \textbf{\small Part D}
\textbf{\small Proof}: \\
\indent We will prove that $\phi(N) = N\prod_{i=1}^{r}(1 - \frac{1}{p_{i}})$ such that
$p_{1}, p_{2}, \ldots ,p_{r}$ are the distinct prime factors of $N$.
\[\phi(N) = \phi((p_{1})^{k_{1}})\phi((p_{2})^{k_{2}}) \ldots \phi((p_{r})^{k_{r}}))\]

Using the formula from part b:
\[\phi(N) = (p_{1}^{k_{1}} - p_{1}^{k_{1} - 1})(p_{2}^{k_{2}} - p_{2}^{k_{2} - 1})
\ldots (p_{r}^{k_{r}} - p_{r}^{k_{r} - 1})\]

\[\phi(N) = p_{1}^{k_{1}}(1 - \frac{1}{p_{1}}) p_{2}^{k_{2}}(1 - \frac{1}{p_{2}})
\ldots  p_{r}^{k_{r}}(1 - \frac{1}{p_{r}})
\]

\[\phi(N) = p_{1}^{k_{1}}p_{2}^{k_{2}} \ldots p_{r}^{k_{r}}(1 - \frac{1}{p_{1}})
\ldots (1 - \frac{1}{p_{r}}) = N(1 - \frac{1}{p_{1}})(1 - \frac{1}{p_{2}})
\ldots (1 - \frac{1}{p_{r}})\]

Thus:
\[\phi(N) = N\prod_{i=1}^{r}(1 - \frac{1}{p_{i}})\]


\noindent \textbf{\small Part E} \\
$\phi(1728) = 576$
$\phi(1575) = 720$
$\phi(889056) = 254016$ \\
Solutions done from formula in part D and checked with the program phi.py
\end{addmargin}

\section*{\large \textbf{Problem 3.8}}
\begin{addmargin}{8pt}
Since Bob chose an $N$ that is too small. Eve can iterate and test all values
to find $p$. This allows Eve to find $p$ and $q$ very easily. Since we know
that $ed \equiv 1 \mod (p-1)(q-1)$, then finding $d$ by iterating through values
will be considered "easy" for Eve. Program used to solve this is in eve.py
Using the program, we conclude that $d = 11629$. \\
\end{addmargin}

\section*{\large \textbf{Problem 3.10}}
\begin{addmargin}{8pt}
\textbf{\small Part A} \\
We know that if N is large that the
$\gcd(k_{1}(p-1)(q-1), k_{2}(p-1)(q-1)) = (p-1)(q-1)$
where $k_{1},k_{2} \in Z$. Because we can find a specific pair of
$d,e$, we can get the k(p-1)(q-1) by $de-1$. Finding $(p-1)(q-1)$ allows us
to find $p + q$ which makes it easy to find a factor of N as we know the
bounds of it in this case. In other words, you can test from values from
$3$ to $q + p$. \\

\noindent \textbf{\small Part B} \\
$p = 5347$,
$q = 7247$ \\

\noindent \textbf{\small Part C} \\
$p=10867$,
$q=20707$ \\

\noindent \textbf{\small Part D} \\
$p=13291$,
$q=97151$ \\
\end{addmargin}

\section*{\large \textbf{Problem 3.13}}
\begin{addmargin}{8pt}
We found $\gcd(e_{1},e_{2})$ is 1.
The equation $e_{1}u + e_{2}v = 1$ from section 3.5 indicate the following:
\[c_{1} * c_{2} = m^{\gcd(e_{1},e_{2})} = m\]
\[m \equiv (c_{1} * c_{2}) \mod N\] \\
Using the numbers given: $m = 13917916680$
\end{addmargin}

\end{document}
